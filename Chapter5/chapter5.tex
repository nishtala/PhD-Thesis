\chapter{Related Work}
\label{chap: relatedwork}


\looseness +1 \lettrine{I}{n} this chapter, we discuss the work related to this
dissertation.  We begin the chapter by discussing the topic of generating performance and
power models for \muc systems (section~\ref{sec: powerperfawaremodels}). Prior work
primarily focused on the problem of building statistical models to detect applications
compute and memory phases to determine the power consumption and the scalability of the
application. In this context, statistical modelling refers to generating time series,
linear regressions, multivariate models, etc., using PMCs
available~\citep{Petrucci:2012:LSE:2387869.2387876, Su:2014:POP:2742155.2742200,
Bellosa:2000:BED:566726.566736, Lewis:2010:CAP:1924920.1924929,
Srinivasan:2011:EIO:1945023.1945032, Rethinagiri:2014:SPE:2555486.2555491, 1598119,
4838819, Patki:2013:EHO:2464996.2465009} on most commercially processors.  It has been
documented in previous studies that a thread's power consumption, performance and energy
consumption can vastly depend on the DVFS of the core/socket and contention for shared
resource (amongst others).  The use of such statistics to determine the causal effect of
performance and power is crucial in data centres to deploy a power cap or to ensure
performance guarantees are met.  For instance, the Intel RAPL~\citep{6270741} module was
designed based on such models to provide the end-users with the capability to control
systems power consumption. 

\looseness +1 Section~\ref{sec: ef technique} discusses the topic of contention-aware
scheduling aimed at maximising energy efficiency in \muc systems.  Prior
work~\citep{Nishtala:2013:ETC:2555754.2555775, Zhuravlev:2013:SES:2498743.2498946,
Blagodurov:2010:CSM:1880018.1880019, Zhuravlev:2012:SST:2379776.2379780,
Fedorova:2007:IPI:1299042.1299108, Knauerhase:2008:UOO:1435611.1436101,
Tam:2007:TCS:1272996.1273004, Tam:2009:RAL:1508284.1508259} has focused predominantly on
the problem of cache contention since this was assumed to the primary, if not the only
source of performance degradation.  In this context, the aforementioned research works
propose using either statistical modelling or heuristics to deal with inter-core or
intra-core contention, and to determine thread's behaviour (compute bound or memory bound)
at runtime, thereby minimising contention for shared resources.

\looseness +1 In our work, we proposed a unified framework to generate models that
estimate the performance and power consumption of workloads at multiple hardware settings.
These models are deployed at runtime to ensure power constraints or performance guarantees
are met.  After extensive experimentation, we suggest that the proposed models can 
estimate performance and power with minimal hardware support for any batch application and
can be deployed on an architecture as they use basic PMCs.  None of the works cited in
Sections~\ref{sec: powerperfawaremodels} and~\ref{sec: ef technique} investigated an
architecture and application-agnostic approach. 

Nevertheless, as the nature of the workloads executing on data centres moving towards
user-centric jobs (i.e., jobs requiring fast response times, typically in the order of
milliseconds measured as tail latency) rather than throughput-oriented (i.e., where
response time of the application does not impact user experience), the plethora of work
addressing scheduling of batch workloads are ineffective as they tend to violate tail
latency requirements~\citep{Lo:2016:IRE:2912575.2882783}. This is because, the metrics to
schedule batch workloads are often focused on instructions per cycle, whereas the
latency-critical workloads are dependent on the interactiveness. In this
context, we introduced an energy efficient (and resource efficient) scheduler for \muc
systems to meet tail latency requirements of interactive workloads. The proposed approach
used a hybrid reinforcement learning approach making it application and architecture
agnostic.  Precisely, our work has shown that we can allocate ``just enough'' resources
for a latency-critical job while meeting their performance targets and the remaining
resources for the batch jobs to maximise throughput.  Section~\ref{sec: hipster-related}
details solutions in the area of cluster-level data centre scheduling of interactive
workloads. 


\section{Performance and Power Modelling} 
\label{sec: powerperfawaremodels}

%Exploring machine learning techniques to predict performance and power (in
%real-environments) at runtime is a well researched area. 

\looseness -1 \textbf{Performance and Power Modelling.} Bellosa~\citep{Bellosa:2000:BED:566726.566736} use PMCs at runtime to build an OS
power-aware policy. Lewis \etal~\citep{Lewis:2010:CAP:1924920.1924929} use a
traditional regression-based methodology and multivariate adaptive regression splines
(MARS) model to present a chaotic time series power activity.  Isci \etal~\citep{1598119}
study compares power phase classifications based on PMCs and proves that PMCs
detect more power phases. Singh \etal~\citep{Singh:2009:RTP:1577129.1577137} propose a
power model using PMCs for an AMD processor. Isci
\etal~\citep{Isci:2003:RPM:956417.956567} show the program behaviour in terms of power
phases. Floyd \etal~\citep{5751940} describe a power and temperature framework for IBM
POWER7 processor using PMCs. Rountree \etal~\citep{6008553} estimate performance
(quantified as IPC) across DVFS states by predicting the total number of leading load
cycles.  Miftakhutdinov \etal~\citep{Miftakhutdinov:2012:PPI:2457472.2457493} predict
performance on simulated architectures based on prefetch and variable memory access
latencies. Spiliopoulos \etal~\citep{6008552} estimate the total time for each LLC miss,
and sums these times to estimate the total time spent in memory, to predict performance at
DVFS states.  Torres \etal~\citep{4536219} propose a memory compression and request
discrimination technique to consolidate workload on minimal number of servers to reduce
wastage of energy. 

Dhiman and Rosing \citep{4838819} build on two of their previous works for DVFS-based
power management algorithm \citep{5514319} and purely
DPM~\citep{Dhiman:2006:DPM:1233501.1233656}. Such methods are now accessible through
Ring-1 of the Linux kernel as C-States. 

%Power limiting techniques are also a well studied area. 
\textbf{Power capping.} Reda \etal~\citep{6266671} introduce a technique to improve
performance by dynamically changing power caps and thread allocations using DVFS. 
Rountree \etal~\citep{6270741} explore the idea of power limits in HPC environments using
RAPL instead of DVFS.  Patki \etal~\citep{Patki:2013:EHO:2464996.2465009} propose an
idea for over-provisioning compute nodes in power-constrained HPC data centres.

\looseness -1 Das \etal~\citep{Das09powercapping}  show a technique to enforce power limits via
forced idleness. Brooks \etal~\citep{Brooks:2000:WFA:339647.339657} shows a power analysis
at a per cycle level for an architecture. Deng \etal~\citep{Deng:2012:CCC:2457472.2457494} propose a simulated system-level
framework with performance and power predictions. Sasaki et
al.~\citep{Sasaki:2013:CPO:2523721.2523732} propose C-3PO, a power manager to maximise
energy efficiency under power constraints. The hardware actuators used to allocate
resources are the DVFS states and number of cores.  

\looseness -1 Cochran~\etal~\citep{Cochran:2011:PCA:2155620.2155641} predict performance with an offline
analysis trained using multi-nomial logistic regression classifier. When a change in
configuration is required at runtime, this classifier returns the best candidate operating
configuration. Petrucci \etal~\citep{Petrucci:2012:LSE:2387869.2387876}, predict
performance in heterogeneous DVFS states on a homogeneous architecture. The work proposes
to build one linear model for every combination of DVFS states using only LLC misses and IPS.  However, prior
works~\citep{Su:2014:POP:2742155.2742200, 10.1109/TC.2012.97, 6008552} have shown that
considering only the LLC misses and MIPS is not a good metric to predict performance at a
very high accuracy. Srinivasan \etal~\citep{Srinivasan:2011:EIO:1945023.1945032}
predict the performance of threads running on heterogeneous cores, that is from one core
type to another, using closed expressions. These expressions, however, do not suffice for
a generic approach. Su \etal~\citep{Su:2014:POP:2742155.2742200} proposes a system-level
performance~\citep{Su:2014:ILL:2643634.2643656} and power model by taking advantage of the
PMCs available in commodity AMD processors to estimate the total
number of leading loads, and in turn, predicts performance and power across DVFS states.  

By contrast to all prior works, Mccullough \etal~\citep{McCullough:2011:EEM:2002181.2002193} state that
linear regression based power models tend to work only in restrictive scenarios and
over-fit based on application types. Our results in prior chapters show that linear
regression models built using a small training dataset do estimate power and performance
for a broad range of workloads with relatively high accuracy and small computational cost. 
    Similar observations were carried out in~\citep{Singh:2009:RTP:1577129.1577137,
    Lewis:2010:CAP:1924920.1924929, Isci:2003:RPM:956417.956567, 10.1109/TC.2012.97}.


In contrast to the aforementioned works, our modelling approach improves in at least three
ways: 

{\small {\circled{1}}} The models are built using basic PMCs available across
all architectures (Intel, AMD and ARM), making it a more generic approach with low
complexity.  

{\small {\circled{2}}} Since our modelling approach is bottom-up, that
is, based on single-core models to predict in \muc environments, it can facilitate for per
core performance and power management.  This is especially useful in multi-node, \muc data
centre consisting of numerous applications with different performance and power
constraints.  

{\small {\circled{3}}} Since our modelling approach can make predictions
at different hardware settings simultaneously.  The prediction approach is an excellent
black-box approach for a single step fine-grained per core power or performance
optimisation problem solver without external power meters or using application
signatures~\citep{Blagodurov:2010:CSM:1880018.1880019}.


\section{Energy Efficient Scheduling} 
\label{sec: ef technique}

%Most IPC-driven algorithms, and some heuristics based algorithms make thread scheduling
%decisions based on based on a quantifiable relative gain metric with regard to running a
%thread on a particular core type.

While SMT-based \muc systems are emerging as the norm for achieving the computing power
necessary~\citep{7029183, 7459368}, it is equally important to map these application to
maximise energy efficiency~\citep{Petrucci:2015:ETA:2724585.2566618,
Porter:2015:MMS:2695583.2687651}. With energy efficient computing emerging as an important
paradigm, recent approaches adopted to using linear programming or heuristics to map
threads-to-cores based on a quantifiable relative gain metric.


Gandhi~\etal~\citep{Gandhi:2010:OAE:1869138.1869264} show a technique to minimise the
product of response time and power costs.
Articles~\citep{Nishtala:2013:ETC:2555754.2555775, Blagodurov:2010:CSM:1880018.1880019, 
Petrucci:2012:LSE:2387869.2387876} describe techniques to schedule workloads to improve
energy efficiency based on the contention for shared resources. 

Becchi~\etal~\citep{Becchi:2006:DTA:1128022.1128029} show a technique for dynamic
thread assignment based on IPC-driven technique on heterogeneous cores by computing speed
up factor based on IPC ratios.  Similarly,
Kumar~\etal~\citep{Kumar:2003:SHM:956417.956569} presented a power aware technique that
dynamically schedules workloads to cores by predicting resource requirements of the
program. For instance, a big core is given to program with high ILP and a small core a
program with low ILP. The approach proposed by Kumar~\etal~allow optimising for different
objectives such as performance and energy.

Sawalha~\etal~\citep{Cong:2012:ESH:2333660.2333737} present a thread scheduling technique
to improve energy efficiency in heterogeneous \muc systems. For each instruction window,
the approach estimates its working set signature based on the instructions executed in a
given phase. By comparing consecutive windows, phases are identified. For the unmarked
phases, the EDP on each core type is determined~\citep{Petrucci:2012:LSE:2387869.2387876}.
For a marked phase, the EDP is assumed to the same as the prior occurrences. The lowest
EDP values from the stored values is chosen for the thread-to-core mapping.

Zhuravlev~\etal~\citep{Zhuravlev:2012:SST:2379776.2379780} and Padmanabha
\etal~\citep{Padmanabha:2013:TBP:2540708.2540746} demonstrate that application signatures
are an effective way to determine applications program context and execution history for
each phase. These signatures are deployed in mappers to schedule workloads to cores can
lead to higher energy efficiency. For instance,
~\citep{Padmanabha:2013:TBP:2540708.2540746} use for scheduling workloads on suitable core
type, whereas~\citep{Zhuravlev:2012:SST:2379776.2379780} identify memory intensive
sections to distribute memory intensity uniformly across domains.
Saez~\etal~\citep{Saez:2010:CSA:1755913.1755929} propose a scheduler to maximise
performance of both single threaded and parallel workloads using the memory intensity of
threads to guide thread assignment decisions.
Koufaty~\etal~\citep{Koufaty:2010:BSH:1755913.1755928} change thread-to-core affinity
measure based thread's memory intensity (LLC misses per committed instruction).

Goiri \etal~\citep{6114408} proposed a method to estimate the amount of solar energy
that might be available and schedule jobs such that the green energy consumption can be
maximised while meeting batch jobs' deadlines.  Similarly, Le \etal~\citep{5598305}
propose algorithms that minimises fossil fuel-based energy consumptions.
Kontorinis \etal~\citep{Kontorinis:2012:MDU:2366231.2337216} have studied the effectiveness of using batteries in data centers. This study shows that using batteries as a primary energy source can lower the capital cost power delivery and the operating costs.


By contrast to the prior approaches, the following
works~\citep{Kumar:2006:CAO:1152154.1152162, Gupta:2011:ASP:1945023.1945026,
Fedorova:2009:MPE:1610252.1610270} analyse microarchitectural characteristics of the
heterogeneous platform to optimise the design area and power or performance efficiency to
help design future \muc systems. Although these works do not provide a scheduling
approach, these works are insightful to enhance capabilities of future \muc and
heterogeneous \muc systems.


\section{QoS Guarantees for Interactive Workloads} 
\label{sec: hipster-related}

Energy efficient cluster computing is focused on meeting deadlines for latency-critical
jobs~\citep{Barroso2003WebArchitecture, 199376, Nishtala2013ScalingFacebook, Atikoglu2012WorkloadStore, villebonnet:hal-01355452} while utilising as little as energy
as possible. This is a major problem in today's data
centres~\citep{Prekas:2015:EPW:2806777.2806848, Kasture2015Rubik, Hoelzle2009TheMachinesb, Wong:2016:PEA:3007787.3001188} and below we summarise the most relevant and
related research.

Mars \etal~\citep{Mars:2011:BIU:2155620.2155650, Yang2013Bubble-flux} detect
at runtime the memory pressure and find the best collocation to avoid negative
interference with latency-critical workloads. They also have a mechanism to detect
negative interference allocations via execution modulation. However such fall-back
mechanism would not adhere to applications like Memcached, as modulations have to be done
at a finer granularity.

\looseness -1 Novakovi\'{c} \etal~\citep{Novakovic2013DeepDive:Environments} identifies and manages performance
interference between VM systems collocated on the same system.
Nathuji \etal~\citep{Nathuji2010Q-clouds} develop a feedback based mechanism to tune resource
assignment to avoid negative interference to collocated VMs systems.
Zhang \etal~\citep{Zhang2013CPI2} enables race-to-finish for low-priority workloads to not have a
deadlock with high priority services.  

%\textbf{Approaches which use latency as performance metric:} 

Petrucci \etal~\citep{Petrucci2015Octopus-Man:Computers} was designed for big.LITTLE
architectures to map workloads on big and small cores at highest DVFS using a feedback
controller in response to changes in measured latency. Lo \etal~\citep{Lo2015Heracles}
uses a feedback controller that exploits collocation of latency-critical and batch
workloads while increasing the resource efficiency of CPU, memory and network as long as
QoS target is met. However, this work is limited to modern Intel architectures due to its
extensive use of cache allocation technology (CAT) and DRAM bandwidth monitor, which are
available from Broadwell processors released after 2015.
Lo \etal~\citep{Lo2014TowardsWorkloads} achieves high CPU energy proportionality for low
latency workloads using fine-grained DVFS techniques.
Vamanan \etal~\citep{Vamanan2015TimeTrader:Search} and Kasture \etal~\citep{Kasture2015Rubik} exploit
request queuing latency variation and apply any available slack from queuing delay to
throughput-oriented workloads to improve energy efficiency.
Delimitrou \etal~\citep{Delimitrou2014Quasar} use runtime classification to predict interference and
collocate workloads to minimise interference. 

Carvalha \etal~\citep{43017} introduce an approach to allocate a subset of the unused
resources for long-term availability SLOs. This methodology deploys time series
forecasting under the premise that availability of resource usage patterns in short jobs.
Nevertheless, this approach does not handle resource allocation effectively as short jobs
do not have certain resource patterns. Furthermore, the approach ignores the unused and
    underused resources caused by time-varying demands in data centre environments.  
 
Tarcil~\citep{Delimitrou:2015:TRS:2806777.2806779} and Firmament~\citep{199390} use
information on the type of resources applications need in a sampling interval to deploy in
distributed scheduler using an analytically-derived sampling framework that provides high
quality resources within a few milliseconds. 


\looseness -1 Wong \etal~\citep{Wong2012KnightShift:Heterogeneity} introduces a server architecture
that couples commercial available compute nodes to adapt the changes in system load and
improve energy proportionality.  Wu \etal~\citep{QiangWuMakingHttps://goo.gl/vJi1kf} (Autoscale) is
for load-balancing a single workload, whereas Hipster could be used for multi-tenant data
centres (different workloads on different nodes). Also, Autoscale cannot exploit
heterogeneity properly. In contrast, at low utilisation, Hipster can use the small cores
for the latency-critical workloads and leave the big cores for batch workloads.
    
Zhou~\etal~\citep{Zhou:2016:GLM:2925426.2926272} proposed a heterogeneous platform-aware
power provisioning system for data centres. The management framework distributes power
from either renewable and non-renewable sources between small and big cores to achieve a
higher energy efficiency while meeting SLO targets.

Violaine~\etal~\citep{villebonnet:hal-01355452} develop a dynamic resource allocation
algorithm which considers each the architectural characteristics of the infrastructure
such as performance, energy consumption, and their turn on/off latency. Using such
information, the framework makes decisions of resource reallocation to ensure that there
exist as little as possible QoS violations for latency-critical workloads while having
potential energy gains.


\looseness -1 Tesauro \etal~\citep{TesauroAAllocation} use an \emph{offline model} based on heuristics
for autonomous resource allocation, which may be limited to specific architectures or
applications. Building a lookup table at runtime is important because applications have
diverse power and performance characteristics which need to be learnt individually (as
shown in Section~\ref{sec:Motivation}). Prior work has previously introduced optimisations
for lookup tables, which include building a priority queue for each load bucket, and
eliminating configurations that seldom occur, and controlling the table size using
function approximations as in~\citep{Vamanan2015TimeTrader:Search,Ipek:2008:SMC:1381306.1382172}.





