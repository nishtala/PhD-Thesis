%Most IPC-driven algorithms, and some heuristics based algorithms make thread scheduling
%decisions based on based on a quantifiable relative gain metric with regard to running a
%thread on a particular core type.

While SMT-based \muc systems are emerging as the norm for achieving the computing power
necessary~\citep{7029183, 7459368}, it is equally important to map these application to
maximise energy efficiency~\citep{Petrucci:2015:ETA:2724585.2566618,
Porter:2015:MMS:2695583.2687651}. With energy efficient computing emerging as an important
paradigm, recent approaches adopted to using linear programming or heuristics to map
threads-to-cores based on a quantifiable relative gain metric.


Gandhi~\etal~\citep{Gandhi:2010:OAE:1869138.1869264} show a technique to minimise the
product of response time and power costs.
Articles~\citep{Nishtala:2013:ETC:2555754.2555775, Blagodurov:2010:CSM:1880018.1880019, 
Petrucci:2012:LSE:2387869.2387876} describe techniques to schedule workloads to improve
energy efficiency based on the contention for shared resources. 

Becchi~\etal~\citep{Becchi:2006:DTA:1128022.1128029} show a technique for dynamic
thread assignment based on IPC-driven technique on heterogeneous cores by computing speed
up factor based on IPC ratios.  Similarly,
Kumar~\etal~\citep{Kumar:2003:SHM:956417.956569} presented a power aware technique that
dynamically schedules workloads to cores by predicting resource requirements of the
program. For instance, a big core is given to program with high ILP and a small core a
program with low ILP. The approach proposed by Kumar~\etal~allow optimising for different
objectives such as performance and energy.

Sawalha~\etal~\citep{Cong:2012:ESH:2333660.2333737} present a thread scheduling technique
to improve energy efficiency in heterogeneous \muc systems. For each instruction window,
the approach estimates its working set signature based on the instructions executed in a
given phase. By comparing consecutive windows, phases are identified. For the unmarked
phases, the EDP on each core type is determined~\citep{Petrucci:2012:LSE:2387869.2387876}.
For a marked phase, the EDP is assumed to the same as the prior occurrences. The lowest
EDP values from the stored values is chosen for the thread-to-core mapping.

Zhuravlev~\etal~\citep{Zhuravlev:2012:SST:2379776.2379780} and Padmanabha
\etal~\citep{Padmanabha:2013:TBP:2540708.2540746} demonstrate that application signatures
are an effective way to determine applications program context and execution history for
each phase. These signatures are deployed in mappers to schedule workloads to cores can
lead to higher energy efficiency. For instance,
~\citep{Padmanabha:2013:TBP:2540708.2540746} use for scheduling workloads on suitable core
type, whereas~\citep{Zhuravlev:2012:SST:2379776.2379780} identify memory intensive
sections to distribute memory intensity uniformly across domains.
Saez~\etal~\citep{Saez:2010:CSA:1755913.1755929} propose a scheduler to maximise
performance of both single threaded and parallel workloads using the memory intensity of
threads to guide thread assignment decisions.
Koufaty~\etal~\citep{Koufaty:2010:BSH:1755913.1755928} change thread-to-core affinity
measure based thread's memory intensity (LLC misses per committed instruction).

Goiri \etal~\citep{6114408} proposed a method to estimate the amount of solar energy
that might be available and schedule jobs such that the green energy consumption can be
maximised while meeting batch jobs' deadlines.  Similarly, Le \etal~\citep{5598305}
propose algorithms that minimises fossil fuel-based energy consumptions.
Kontorinis \etal~\citep{Kontorinis:2012:MDU:2366231.2337216} have studied the effectiveness of using batteries in data centers. This study shows that using batteries as a primary energy source can lower the capital cost power delivery and the operating costs.


By contrast to the prior approaches, the following
works~\citep{Kumar:2006:CAO:1152154.1152162, Gupta:2011:ASP:1945023.1945026,
Fedorova:2009:MPE:1610252.1610270} analyse microarchitectural characteristics of the
heterogeneous platform to optimise the design area and power or performance efficiency to
help design future \muc systems. Although these works do not provide a scheduling
approach, these works are insightful to enhance capabilities of future \muc and
heterogeneous \muc systems.
