\looseness -1 \textbf{Performance and Power Modelling.} Bellosa~\citep{Bellosa:2000:BED:566726.566736} use PMCs at runtime to build an OS
power-aware policy. Lewis \etal~\citep{Lewis:2010:CAP:1924920.1924929} use a
traditional regression-based methodology and multivariate adaptive regression splines
(MARS) model to present a chaotic time series power activity.  Isci \etal~\citep{1598119}
study compares power phase classifications based on PMCs and proves that PMCs
detect more power phases. Singh \etal~\citep{Singh:2009:RTP:1577129.1577137} propose a
power model using PMCs for an AMD processor. Isci
\etal~\citep{Isci:2003:RPM:956417.956567} show the program behaviour in terms of power
phases. Floyd \etal~\citep{5751940} describe a power and temperature framework for IBM
POWER7 processor using PMCs. Rountree \etal~\citep{6008553} estimate performance
(quantified as IPC) across DVFS states by predicting the total number of leading load
cycles.  Miftakhutdinov \etal~\citep{Miftakhutdinov:2012:PPI:2457472.2457493} predict
performance on simulated architectures based on prefetch and variable memory access
latencies. Spiliopoulos \etal~\citep{6008552} estimate the total time for each LLC miss,
and sums these times to estimate the total time spent in memory, to predict performance at
DVFS states.  Torres \etal~\citep{4536219} propose a memory compression and request
discrimination technique to consolidate workload on minimal number of servers to reduce
wastage of energy. 

Dhiman and Rosing \citep{4838819} build on two of their previous works for DVFS-based
power management algorithm \citep{5514319} and purely
DPM~\citep{Dhiman:2006:DPM:1233501.1233656}. Such methods are now accessible through
Ring-1 of the Linux kernel as C-States. 

%Power limiting techniques are also a well studied area. 
\textbf{Power capping.} Reda \etal~\citep{6266671} introduce a technique to improve
performance by dynamically changing power caps and thread allocations using DVFS. 
Rountree \etal~\citep{6270741} explore the idea of power limits in HPC environments using
RAPL instead of DVFS.  Patki \etal~\citep{Patki:2013:EHO:2464996.2465009} propose an
idea for over-provisioning compute nodes in power-constrained HPC data centres.

\looseness -1 Das \etal~\citep{Das09powercapping}  show a technique to enforce power limits via
forced idleness. Brooks \etal~\citep{Brooks:2000:WFA:339647.339657} shows a power analysis
at a per cycle level for an architecture. Deng \etal~\citep{Deng:2012:CCC:2457472.2457494} propose a simulated system-level
framework with performance and power predictions. Sasaki et
al.~\citep{Sasaki:2013:CPO:2523721.2523732} propose C-3PO, a power manager to maximise
energy efficiency under power constraints. The hardware actuators used to allocate
resources are the DVFS states and number of cores.  

\looseness -1 Cochran~\etal~\citep{Cochran:2011:PCA:2155620.2155641} predict performance with an offline
analysis trained using multi-nomial logistic regression classifier. When a change in
configuration is required at runtime, this classifier returns the best candidate operating
configuration. Petrucci \etal~\citep{Petrucci:2012:LSE:2387869.2387876}, predict
performance in heterogeneous DVFS states on a homogeneous architecture. The work proposes
to build one linear model for every combination of DVFS states using only LLC misses and IPS.  However, prior
works~\citep{Su:2014:POP:2742155.2742200, 10.1109/TC.2012.97, 6008552} have shown that
considering only the LLC misses and MIPS is not a good metric to predict performance at a
very high accuracy. Srinivasan \etal~\citep{Srinivasan:2011:EIO:1945023.1945032}
predict the performance of threads running on heterogeneous cores, that is from one core
type to another, using closed expressions. These expressions, however, do not suffice for
a generic approach. Su \etal~\citep{Su:2014:POP:2742155.2742200} proposes a system-level
performance~\citep{Su:2014:ILL:2643634.2643656} and power model by taking advantage of the
PMCs available in commodity AMD processors to estimate the total
number of leading loads, and in turn, predicts performance and power across DVFS states.  

By contrast to all prior works, Mccullough \etal~\citep{McCullough:2011:EEM:2002181.2002193} state that
linear regression based power models tend to work only in restrictive scenarios and
over-fit based on application types. Our results in prior chapters show that linear
regression models built using a small training dataset do estimate power and performance
for a broad range of workloads with relatively high accuracy and small computational cost. 
    Similar observations were carried out in~\citep{Singh:2009:RTP:1577129.1577137,
    Lewis:2010:CAP:1924920.1924929, Isci:2003:RPM:956417.956567, 10.1109/TC.2012.97}.


In contrast to the aforementioned works, our modelling approach improves in at least three
ways: 

{\small {\circled{1}}} The models are built using basic PMCs available across
all architectures (Intel, AMD and ARM), making it a more generic approach with low
complexity.  

{\small {\circled{2}}} Since our modelling approach is bottom-up, that
is, based on single-core models to predict in \muc environments, it can facilitate for per
core performance and power management.  This is especially useful in multi-node, \muc data
centre consisting of numerous applications with different performance and power
constraints.  

{\small {\circled{3}}} Since our modelling approach can make predictions
at different hardware settings simultaneously.  The prediction approach is an excellent
black-box approach for a single step fine-grained per core power or performance
optimisation problem solver without external power meters or using application
signatures~\citep{Blagodurov:2010:CSM:1880018.1880019}.
