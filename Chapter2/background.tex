\section{Dynamic Power and Performance Management} 
\label{section: Background} 

Currently available modern operating systems control the power dissipated by dynamically
controlling the power state in which the processor is at runtime.  Most operating systems,
provide an interface to monitor the processor's power state to communicate and control
with the hardware subsystem. For instance, on \textsf{Linux}, Advanced Configuration and
Power Interface (ACPI) defines a protocol to control these power states. In this section,
we give a background of the hardware settings used in this thesis.

\paragraph*{Dynamic Power Management (DPM).} DPM schemes are ingrained as an important
means to reduce power consumption in data centres having hundreds or thousands of cores,
each of them with multiple power management features. We focus on the primary DPM features
available our architectures: processor performance states (DVFS states), processor power states
(C-States), clamp states (Cl-States) and core consolidation.
DVFS states~\citep{Eric2009TheSearch, Su:2014:POP:2742155.2742200, Cochran:2011:PCA:2155620.2155641, Huang2000AManagement,Isci2006AnBudget, 6008553,  Singh2009RealCounters} and
Cl-States~\citep{Olsen:2006:PEP:1137248.1137532} have been used in numerous previous works
to provide fine-grained low-level support for a reasonable performance-power trade-off.  

\paragraph*{Processor performance states (P-States/DVFS).} Each DVFS state represents a
specified voltage and clock frequency of the cores~\citep{acpikernel} and is controlled by
the ACPI.  The frequency and voltage for
each state is described in the Differentiated System Description Table (DSDT).  The higher
numbered DVFS states correspond to higher clock frequency and higher voltage, thereby
delivering a higher performance subjected to the scalability of the workload.

The changes in DVFS state are governed using the ACPI \textsf{CPUFreq} governor using the
\textsf{acpi\_cpufreq} driver.  The default governor available using the driver as
mentioned above is \textsf{ondemand}~\citep{ondemand}, which allows the kernel to control
the core's DVFS state based on the ``load'' on the processor. However, the governor
\textit{userspace} lets users to control the core's DVFS state, via the \textsf{sysfs}
file system, manually irrespective of the ``load''.

\looseness -1 \paragraph*{Processor power states (C-States).} Modern \muc processors
support multiple C-States, also known as ``Sleep States''. These states allow various
internal modules to be turned off. Higher-numbered C-States indicate a ``deeper'' sleep
state with more internal modules disabled resulting in a lower power consumption, but
entails a longer wake up period (up to \SI{1024}{\nano\second} on an Intel machine).  At
a given C-State, all DVFS states have same power consumption when
idle~\citep{Olsen:2006:PEP:1137248.1137532}.  The choice of C-States is architecture and
processor dependent. For example, Intel Sandy Bridge supports C0, C1, C3, C6 and C7 states
and they are accessible through the \textsf{sysfs} file system, similar to the DVFS states.

\looseness -1  This technology on Intel and AMD processors is coined as
``C-States''~\citep{Intel, Schone:2015:WLP:2766527.2766547}, and on ARM processors as
``CPUidleness''~\citep{maillistperfmon}. For instance, most Intel processors support C0,
C1, C3, C6 and C7 states.  The residency period indicates the duration of time a given
core is present at a given C-State and can be controlled using the model-specific
registers (MSR).  The default time quantum for \textit{entering} a C-State is
\SI{1024}{\nano\second}.  

\paragraph*{Clamp States (Cl-States).} Although C-States are a effective power reduction
mechanisms, they are used to reduce power consumption only when the system is idle.
On the other hand, \textit{Cl-States} introduce synchronised idle injections across online
CPU threads to enable forced and controllable C-State residency.  The changes in Cl-States
are governed by ACPI using the \textsf{intel\_powerclamp} driver~\citep{powerclamp}. 


\nomenclature[z-ACPI]{ACPI}{Advanced Configuration and Power Interface}
\nomenclature[z-DSDT]{DSDT}{Differentiated State Description Table} 


