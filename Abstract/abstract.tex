\begin{abstract} 
    
    \looseness +1 \lettrine{I}{n} 2013, U.S. data centres accounted for \SI{2.2}{\percent}
    of the country's total electricity consumption, a figure that is projected to increase
    rapidly over the next decade.  A significant proportion of power consumed within a
    data centre is attributed to the servers, and a large percentage of that is wasted as
    workloads compete for shared resources.  Many data centres host interactive workloads
    (e.g., web search or e-commerce), for which it is critical to meet user expectations
    and user experience, called Quality of Service (QoS).  There is also a wish to run
    both interactive and batch workloads on the same infrastructure to increase cluster
    utilisation and reduce operational costs and total energy consumption. Although much
    work has focused on the impacts of shared resource contention, it still remains a
    major problem to maintain QoS for both interactive and batch workloads. The goal of
    this thesis is twofold. First, to investigate how, and to what extent, resource
    contention has an effect on throughput and power of batch workloads via modelling.
    Second, we introduce a scheduling approach to determine on-the-fly the best
    configuration to satisfy the QoS for latency-critical jobs on any architecture.
     
    \looseness -1 To achieve the above goals, we first propose a modelling technique to
    estimate server performance and power at runtime called \emph{Runtime Estimation of
    Performance and Power} (\textbf{REPP}).  REPP's goal is to allow administrators'
    control on power and performance of processors.  REPP achieves this goal by estimating
    performance and power at multiple hardware settings (dynamic frequency and voltage
    states (DVFS), core consolidation and idle states) and dynamically sets these settings
    based on user-defined constraints.  The hardware counters required to build the models
    are available across architectures, making it architecture agnostic.
    
    \looseness -1 We also argue that traditional modelling and scheduling strategies are
    ineffective for interactive workloads. To manage such workloads, we propose
    \textbf{Hipster} that combines both a heuristic, and a reinforcement learning
    algorithm to manage interactive workloads.  Hipster's goal is to improve resource
    efficiency while respecting the QoS of interactive workloads.  Hipster achieves its
    goal by exploring the \muc system and DVFS. To improve utilisation and make the best
    usage of the available resources, Hipster can dynamically assign remaining cores to
    batch workloads without violating the QoS constraints for the interactive workloads. 

   We implemented REPP and Hipster in real-life platforms, namely \SIadj{64}{\bit}
    commercial (Intel SandyBridge and AMD Phenom~\rom{2} X4 B97) and experimental hardware
    (ARM big.LITTLE Juno R1).  After obtaining extensive experimental results, we have
    shown that REPP successfully estimates power and performance of several
    single-threaded and multiprogrammed workloads. The average errors on Intel, AMD and
    ARM architectures are, respectively, \SI{7.1}{\percent}, \SI{9.0}{\percent},
    \SI{7.1}{\percent} when predicting performance, and \SI{8.1}{\percent},
    \SI{6.5}{\percent}, \SI{6.0}{\percent} when predicting power.  Similarly, we show that
    when compared to prior work, Hipster improves the QoS guarantee for Web-Search from
    \SI{80}{\percent} to \SI{96}{\percent}, and for Memcached from \SI{92}{\percent} to
    \SI{99}{\percent}, while reducing the energy consumption by up to \SI{18}{\percent} on
    the ARM architecture.  
    
\end{abstract}
