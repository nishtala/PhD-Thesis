\ifsetCustomMargin
  \RequirePackage[left=37mm,right=30mm,top=35mm,bottom=30mm]{geometry}
  \setFancyHdr % To apply fancy header after geometry package is loaded
\fi

\raggedbottom

\ifsetCustomFont
\fi

\usepackage{varwidth}

\usepackage[noend]{algpseudocode}
\usepackage{algorithm}

\usepackage{wrapfig}
%\usepackage{algpseudocode}
\newcommand{\Break}{\State \textbf{break} }
\algblockdefx[Loop]{Loop}{EndLoop}[1][]{\textbf{Loop} #1}{\textbf{End Loop}}


\usepackage{varioref} %force table position http://tex.stackexchange.com/questions/9485/how-to-fix-table-position

\let\oldReturn\Return
\renewcommand{\Return}{\State\oldReturn}
\newcommand{\vars}{\texttt}
\newcommand{\func}{\textrm}

\algrenewcommand\alglinenumber[1]{\footnotesize #1}

\usepackage[font={small},labelfont={bf},margin=15pt, labelsep=space, tableposition=top]{caption} % load after babel. also replaces hypcap.


\usepackage{float}
\usepackage{subcaption}
\usepackage[percent]{overpic}

\usepackage{colortbl,array,booktabs} % For professional looking tables
\usepackage{multirow}

\setlength\heavyrulewidth{1.5pt} %toprule width

% *********************************** SI Units *********************************
\usepackage{siunitx} % use this package module for SI units
\sisetup{per-mode=symbol}
\sisetup{binary-units=true}
\DeclareBinaryPrefix\Kilo{K}{10}
\providecommand{\e}[1]{\ensuremath{\times 10^{#1}}}
\newcommand{\sn}[2]{\ensuremath{{#1}\times 10^{#2}}}
\newcommand{\SIadj}[2]{\SI[number-unit-product={\text{-}}]{#1}{#2}}

\usepackage[usenames,dvipsnames]{xcolor}
\usepackage{hyperref}
\hypersetup{
  colorlinks=True,
  citecolor=Green,
  linkcolor=Red,
  urlcolor=Blue,
  linktoc=all,
  }

\ifuseCustomBib
   \RequirePackage[square, sort, numbers, authoryear, backref=true,backrefstyle=none]{biblatex} % CustomBib
\fi

\renewcommand{\bibname}{References}

\newenvironment{romanpages}{
  \setcounter{page}{1}
  \renewcommand{\thepage}{\roman{page}}}
{\newpage\renewcommand{\thepage}{\arabic{page}}}

\usepackage[most]{tcolorbox}

\usepackage{collectbox}

\makeatletter
\newcommand{\mybox}{%
    \collectbox{%
        \setlength{\fboxsep}{1pt}%
        \fbox{\BOXCONTENT}%
    }%
}
\makeatother

\setcounter{secnumdepth}{2}
\setcounter{tocdepth}{2}


\usepackage[autostyle]{csquotes} 

\usepackage{titlesec}  % used to setup the titleformat %Bug in titleformat
\usepackage{epigraph}  % for using parts

\usepackage{etoolbox}

\makeatletter
\patchcmd{\ttlh@hang}{\parindent\z@}{\parindent\z@\leavevmode}{}{}
\patchcmd{\ttlh@hang}{\noindent}{}{}{}
\makeatother

\newcommand{\lofchap}[1]{%
    \addtocontents{lof}{\protect\addvspace{-10pt}}% Undo the unconditional 10pt skip from latex
    \addcontentsline{lof}{lofchapter}{\protect\numberline{\thechapter}#1}%
}
\iffalse
\makeatletter
\ifthenelse{\boolean{@linedheaders}}%
    {% lines above and below, number right
    \titleformat{\chapter}[display]%             
        {\relax}
        {\raggedleft{\color{halfgray}\chapterNumber\thechapter} \\ }
        {0pt}%
        {\color{green!70!black}\huge\titlerule\vspace*{.9\baselineskip}\raggedright\spacedallcaps}% change color and size here 
        [\normalcolor\normalsize\vspace*{.8\baselineskip}\titlerule]%
    }
    {% something like Bringhurst  
    \titleformat{\chapter}[display]%
        {\relax}
        {\mbox{}\oldmarginpar{\vspace*{-3\baselineskip}\color{halfgray}\chapterNumber\thechapter}}
        {0pt}%
        {\color{green!70!black}\huge\raggedright\spacedallcaps}% change color and size here
        [\normalcolor\normalsize\vspace*{.8\baselineskip}\titlerule]% 
    }
\makeatother  
\fi

\titleformat{\chapter}[display]
    {\bfseries\Large}
    {\filleft\MakeUppercase{\chaptertitlename} \Huge\thechapter}
    %{\filright\MakeUppercase{\chaptertitlename} \Huge\thechapter}
    {1ex}
    {\titlerule\vspace{1.5ex}\filleft}
    [\vspace{1.5ex}\titlerule] 


%\usepackage{hyperref}


%%CIRLCED
\usepackage{tikz}
\newcommand*\circled[1]{\tikz[baseline=(char.base)]{
            \node[shape=circle,draw,inner sep=2pt] (char) {#1};}}


%ADDED THIS
\usetikzlibrary{tikzmark,fit, arrows}
\usetikzlibrary{positioning}
\usetikzlibrary{decorations.pathreplacing}

\tikzset{
  invisible/.style={opacity=0},
  visible on/.style={alt={#1{}{invisible}}},
  alt/.code args={<#1>#2#3}{%
    \alt<#1>{\pgfkeysalso{#2}}{\pgfkeysalso{#3}} % \pgfkeysalso doesn't change the path
  },
}

\usepackage{pgf}

\usepackage{pgfplots}
        


%% TODONOTES
\usepackage{todonotes}
%\usepackage[disable]{todonotes} % TO DISABLE

\newcommand{\rnline}[1]{\todo[author=nishtala,size=\small,inline,color=green!40]{#1}}
\newcommand{\rn}[1]{\todo[color=blue!40]{#1}}


\usepackage{cite}

\usepackage{rotating}  %Used for rotating tables \begin{sidewaystable}\end{sidewaystable}
%\usepackage{makecell}

\setcounter{secnumdepth}{3} % to ensure subsub sections have numbering

\usepackage{physics} % For abs
\usepackage{relsize}

\usepackage{amsmath}

%\usepackage[percent]{overpic} %for adding annotations on the side
\usepackage{stackengine,graphicx}

%%%LETTRINE

\usepackage{lettrine}  % dropped capitals. doesn't work so well with doublespacing?
    \renewcommand{\DefaultLhang}{0.1}
%    \renewcommand{\DefaultOptionsFile}{\lettrineconffile}
%    \renewcommand{\LettrineTextFont}{\rmfamily}


%TEXT formatting for figures

\newcommand{\captitle}[1]{\textbf{#1}}
% Use extra bold for caption letters, for easier spotting...
\newcommand{\capl}[1]{{\fontseries{eb}\selectfont{#1}}}

%\usepackage[babel,final]{microtype} % must come after babel. uses draft/final. [selected]?

\usepackage{amsfonts}
\usepackage{amssymb}

\usepackage{fancyhdr}

\makeatletter
\newcommand*{\rom}[1]{\expandafter\@slowromancap\romannumeral #1@} %necessary to write AMD Opteron II
\makeatother

\usepackage{microtype}

%Custom itemize style
\usepackage{bbding}

%FANCY text figures
\renewcommand{\thefigure}{%
\ifnum\value{chapter}>0 \oldstylenums{\arabic{chapter}}.\fi\oldstylenums{\arabic{figure}}}

\renewcommand{\thetable}{%
\ifnum\value{chapter}>0 \oldstylenums{\arabic{chapter}}.\fi\oldstylenums{\arabic{table}}}

\renewcommand{\oldstylenums}[1]{{\fontfamily{pplj}\selectfont #1}}



\newcommand{\ra}[1]{\renewcommand{\arraystretch}{#1}} % to change widtth between rows in booktabs

%FOR EACH
\algnewcommand\algorithmicforeach{\textbf{for each}}
\algdef{S}[FOR]{ForEach}[1]{\algorithmicforeach\ #1\ \algorithmicdo}

%Function call
\iffalse
\makeatletter
\renewcommand{\Function}[2]{%
  \csname ALG@cmd@\ALG@L @Function\endcsname{#1}{#2}%
  \def\jayden@currentfunction{#1}%
}
\newcommand{\funclabel}[1]{%
  \@bsphack
  \protected@write\@auxout{}{%
    \string\newlabel{#1}{{\jayden@currentfunction}{\thepage}}%
  }%
  \@esphack
}
\makeatother
\fi

\algdef{SE}[PROCEDURE]{Procedure}{EndProcedure}%
   [2]{\algorithmicprocedure\ \textproc{#1}\ifthenelse{\equal{#2}{}}{}{(#2)}}%
   {\algorithmicend\ \algorithmicprocedure}%
\algdef{SE}[FUNCTION]{Function}{EndFunction}%
   [2]{\algorithmicfunction\ \textproc{#1}\ifthenelse{\equal{#2}{}}{}{(#2)}}%
   {\algorithmicend\ \algorithmicfunction}%

\algtext*{EndFunction}% Remove "end function" text



%Approximately equal
\usepackage{scalerel}
\def\apeqA{\SavedStyle\sim}
\def\apeq{\setstackgap{L}{\dimexpr.5pt+1.5\LMpt}\ensurestackMath{%
  \ThisStyle{\mathrel{\Centerstack{{\apeqA} {\apeqA} {\apeqA}}}}}}

%PARTS
\makeatletter
\titleformat{\part}[display]
  {\Huge\scshape\filright}
  {\partname~\thepart:}
  {20pt}
  {\thispagestyle{epigraph}}
\makeatother
\setlength\epigraphwidth{.6\textwidth}

%PIE CHART BEGIN
\usepackage{calc}
\usepackage{ifthen}

\newcommand{\slice}[4]{
  \pgfmathparse{0.5*#1+0.5*#2}
  \let\midangle\pgfmathresult

  % slice
  \draw[thick,fill=black!10] (0,0) -- (#1:1) arc (#1:#2:1) -- cycle;

  % outer label
  \node[label=\midangle:#4] at (\midangle:1) {};

  % inner label
  \pgfmathparse{min((#2-#1-10)/110*(-0.3),0)}
  \let\temp\pgfmathresult
  \pgfmathparse{max(\temp,-0.5) + 0.8}
  \let\innerpos\pgfmathresult
  \node at (\midangle:\innerpos) {#3};
}

\DeclareCaptionLabelFormat{andtable}{#1~#2  \&  \tablename~\thetable} % having table and figure with same caption for "scalability of repp-h"


%\DeclareDividedList{lof}
%\DeclareDividedList{lot}

%\printbibliography[heading=bibnumbered]  
\usepackage{tocbibind} % put things in the table of contents.

\usepackage{scrextend} 

\addtokomafont{labelinglabel}{\sffamily}

\usepackage{enumitem}

\usepackage{gensymb} % to write degree symbol

\usepackage{arydshln} % for vertical and horizontal dash lines

\usepackage[version=3]{mhchem} % to write co2 in intro

%for eurosign

\usepackage{eurosym}
%\font\titlerm=lmr12 
\usepackage{lmodern}
%\usepackage{mathpazo} % add possibly `sc` and `osf` options
%\usepackage{eulervm}
\usepackage{newpxmath}
\usepackage[scaled=0.95]{inconsolata}

%\linepenalty=50

%PIE CHART
\definecolor{rosso}{RGB}{220,57,18}
\definecolor{giallo}{RGB}{255,153,0}
\definecolor{blu}{RGB}{102,140,217}
\definecolor{verde}{RGB}{16,150,24}
\definecolor{viola}{RGB}{153,0,153}

\makeatletter

\tikzstyle{chart}=[
    legend label/.style={font={\scriptsize},anchor=west,align=left},
    legend box/.style={rectangle, draw, minimum size=5pt},
    axis/.style={black,semithick,->},
    axis label/.style={anchor=east,font={\tiny}},
]

\tikzstyle{bar chart}=[
    chart,
    bar width/.code={
        \pgfmathparse{##1/2}
        \global\let\bar@w\pgfmathresult
    },
    bar/.style={very thick, draw=white},
    bar label/.style={font={\bf\small},anchor=north},
    bar value/.style={font={\footnotesize}},
    bar width=.75,
]

\tikzstyle{pie chart}=[
    chart,
    slice/.style={line cap=round, line join=round, very thick,draw=white},
    pie title/.style={font={\bf}},
    slice type/.style 2 args={
        ##1/.style={fill=##2},
        values of ##1/.style={}
    }
]

\pgfdeclarelayer{background}
\pgfdeclarelayer{foreground}
\pgfsetlayers{background,main,foreground}


\newcommand{\pie}[3][]{
    \begin{scope}[#1]
    \pgfmathsetmacro{\curA}{90}
    \pgfmathsetmacro{\r}{1}
    \def\c{(0,0)}
    \node[pie title] at (90:1.3) {#2};
    \foreach \v/\s in{#3}{
        \pgfmathsetmacro{\deltaA}{\v/100*360}
        \pgfmathsetmacro{\nextA}{\curA + \deltaA}
        \pgfmathsetmacro{\midA}{(\curA+\nextA)/2}

        \path[slice,\s] \c
            -- +(\curA:\r)
            arc (\curA:\nextA:\r)
            -- cycle;
        \pgfmathsetmacro{\d}{max((\deltaA * -(.5/50) + 1) , .5)}

        \begin{pgfonlayer}{foreground}
        \path \c -- node[pos=\d,pie values,values of \s]{$\v\%$} +(\midA:\r);
        \end{pgfonlayer}

        \global\let\curA\nextA
    }
    \end{scope}
}



\newcommand{\legend}[2][]{
    \begin{scope}[#1]
    \path
        \foreach \n/\s in {#2}
            {
                  ++(0,-10pt) node[\s,legend box] {} +(5pt,0) node[legend label] {\n}
            }
    ;
    \end{scope}
}

%%%END PIE CHART

%MACROS
\newcommand*{\het}{heterogeneous } 

\newcommand*{\muc}{multicore }

\newcommand*{\etal}{\textit{et al.}} % "et. al. should be et al. morons!"

\newcommand*{\archs}{architectures }
\newcommand\tab[1][1cm]{\hspace*{#1}}


\newcommand*{\ninefive}{\textrm{95\textsuperscript{$\mathit{th}$}} }
\newcommand*{\ninezero}{\textrm{90\textsuperscript{$\mathit{th}$}} }

\newcommand*{\ninenine}{\textrm{99\textsuperscript{$\mathit{th}$}} }

\newcommand*{\mathcalS}{$\mathcal S$ }

\newcommand*{\mathcalP}{$\mathcal P$ }

\newcommand*{\dcs}{data centres}
