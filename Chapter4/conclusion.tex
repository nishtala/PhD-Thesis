\section{Conclusion}
\label{sec: conclusion-hipster}

\looseness -1 We propose Hipster, a hybrid scheme that combines heuristics and
reinforcement learning to manage heterogeneous cores with DVFS control for improved energy
efficiency. We show that Hipster performs well across workloads and interactively adapts
the system by learning from the QoS/power/performance history to best map workloads to the
heterogeneous cores and adjust their DVFS settings.  When only latency-critical workloads
are running in the system, Hipster reduces energy consumption by \SI{13}{\percent} in
comparison to prior work. In addition, to improve resource efficiency in shared data
centres by running both latency-critical and batch workloads on the same system, Hipster
improves batch workload throughput by $2.3\times$ compared to a static and conservative
policy, while meeting the QoS targets for the latency-critical workloads.
