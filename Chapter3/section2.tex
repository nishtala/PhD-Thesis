\chapter{Power and Performance Models for Server Systems} 

\label{section: REPP-H} 


\lettrine{I}{ncreasing} visibility of server architecture heterogeneity in data centres
raises a demand for a scheduling/modelling approach that allocates precise amount of
resources to workloads based on their power and performance
requirements~\citep{Mars2013Whare-map, Nishtala:HPCA, Petrucci2015Octopus-Man:Computers,
Kasture2015Rubik, Lo2015Heracles}. In order to exploit the heterogeneity, there is a need
for a simple, application agnostic framework to estimate application demands
\emph{on-the-fly}. Such an approach can be deployed to adapt and control the resource
allocation to optimise performance and power on each architecture. In this chapter, we
demonstrate a method to control and adapt systems power consumption and performance demand
based on external constraints by modelling the performance and power at multiple hardware
control settings in a \muc environment.

%and adapting them at runtime based on applications behaviour. 


%These SLA agreements can be defined as PUE of the data centre, maximum power consumption,
%and delivering a minimum throughput.  For instance, PUE is an extensively used metric in
%Google data centres~\citep{4544393, Google:PUE}.  Nevertheless such architectures raise
%interesting challenges on cluster-level scheduling in meeting any of the aforementioned
%SLA agreements. 

%In the previous section, we introduced procedure to build performance and power models
%offline for a single-core. Then, we evaluated the single-core models offline and online
%on three different architectures, and have shown that our methodology predicts
%performance and power with less than \SI{10}{\percent} error. 

Specifically, we extend the single-core models built (Section~\ref{sec: REPP}) for a
\emph{single-core} to \emph{multicore} environments and validate with multiple
multiprogrammed workloads in heterogeneous architectures (REPP- Heterogeneous:
\textbf{REPP-H}). REPP-H can be deployed across architectures as the PMCs selected are
basic and available on most commercial platforms.  REPP-H also provides a methodology to
meet performance and power constraints in \muc environments with high accuracy.  

%The subsequent sections of this chapter introduce the multicore modelling technique, the
%evaluation of multicore models without constraints and with constraints.  




\section{Conclusion} 
\label{subsection: conclusion}


